%% !TEX root = manual.tex

\newcommand{\openTable}{\begin{tabular}{| p{5cm} | p{2cm} | p{3cm} | p{8cm} |}}
\newcommand{\paramType}[1]{\newline (#1)}
\hyphenpenalty=10000
\pagestyle{empty}

\chapter{Detailed Parameter Listings}
\label{chapter:parameters}
The following chapter is organized by parameter namespaces. Tables in each namespace are organized as
\def\arraystretch{1.5}%  1 is the default, change whatever you need

\openTable
\hline
Name (type) & Default if not given & Allowed \newline Values & Description \\
\hline
\end{tabular}

which lists the possible parameter names, allowed values, and brief descriptions.
More detailed descriptions of particular parameter values are found in the documentation in previous chapters.

The allowed parameter types are:

\begin{tabular}{| l | l |}
\hline
int & Any integer \\
\hline
long & Any integer value, but guaranteed not to overflow for long integers \\
\hline
bool & Either ``true'' or ``false'' as lowercase string \\
\hline
time & Any valid float value followed by time units (s,ms,us,ns,ps) \\
\hline
freq & Any valid float value followed by frequency units (Hz, MHz, GHz) \\
\hline
bandwidth & Any valid float value followed by bandwidth units (b/s, B/s, Mb/s, MB/s, etc) \\
\hline
byte length & Any positive integer followed by length units (B, KB, MB, GB,TB) \\
\hline
string & An arbitrary string \\
\hline
vector of X & A vector of type X with entries separated by spaces \\
\hline
filepath & A valid filepath to an \emph{existing} file, either absolute or relative \\
\hline
quantiy & A catch-all for a quantity with units. Any of frequency, bandwidth, byte length, or time can be given \\
\hline
\end{tabular}
\section{Global namespace}
\openTable
\hline
sst\_nthread \paramType{int} & 1 & Positive int & Only relevant for multi-threading. Specifying more threads than cores can lead to deadlock. \\
\hline
timestamp\_resolution \paramType{time} & 1ps & & Specifies the length of time occupied by 1 timestamp tick - the smallest resolvable time difference. Numerical stability depends on this parameter matching the time scales of the simulation. \\
\hline
serialization\_buffer\_size \paramType{byte length} & 512B & & Size to allocate for buffers in a pool of serialization buffers. This should set be large enough to handle serialization of most messages, but not so large that significant space is wasted. \\
\hline
cpu\_affinity \paramType{vector of int} & No default & Invalid cpu IDs give undefined behavior & When in multi-threading, specifies the list of core IDs that threads will be pinned to. \\
\hline
\end{tabular}

\section{Namespace ``topology''}
\openTable
\hline
geometry \paramType{vector of int} & No default & See Topology section & Geometry configuration of the topology. For details of the keyword, users should refer to Section \ref{chapter:topologies} \\
\hline
name \paramType{string} & No default & torus, dragonfly, fat\_tree, butterfly, crossbar, simple\_fat\_tree & The name of the topology to build. For details, see Section \ref{chapter:topologies} \\
\hline 
seed \paramType{long} & System time & & If no value given, random numbers for topology will be generated from system time \\
\hline
concentration \paramType{int} & 1 & Positive int & The number of nodes per network switch. For indirect networks, this is the number of nodes per leaf switch. \\
\hline
num\_leaf\_switches \paramType{int} & No default & Positive int & Only relevant for fat trees. This is the number of switches at the lowest level of the tree that are connected to compute nodes. \\
\hline
tapering \paramType{vector of double} & No default & Values 0-1 & For tapering of a fat tree, this specifies a tapering fraction per level of the fat tree.  This specifies a bandwidth multiplier for each layer of switch-to-switch links. \\
\hline
k \paramType{int} & No default & int >= 2 & The branching fraction of a fat tree. k=2 is a binary tree. k=4 is a quad-tree. \\
\hline
group\_connections \paramType{int} & No default & Positive int & For dragonfly, the number of intergroup connections on each switch in a Dragonfly group \\
\hline
redundant \paramType{vector of int} & vector of 1's & Positive ints & For Cartesian topologies (hypercube, dragonfly, torus) this specifies a bandwidth (redundancy) multiplier for network links in each dimension. \\
\hline
\end{tabular}

\section{Namespace ``node''}
\openTable
\hline
services \paramType{vector of strings} & Empty & Valid service names & For  details, see section on distributed services in developer's manual. Advanced feature. \\
\hline
\end{tabular}
\subsection{Namespace ``node.nic''}
\openTable
\hline
negligible\_size \paramType{byte length} & 256B & & Messages (flows) smaller than size will not go through detailed congestion modeling. They will go through a simple analytic model to compute the delay. \\
\hline
\end{tabular}
\subsubsection{Namespace ``node.nic.traffic\_matrix''}
\openTable
\hline
fileroot \paramType{string} & No default & & 
  The path of a file to be created (or overwritten) where results from the statistics will be placed. 
 \\
\hline 
type \paramType{string} & ascii & ascii, png & Whether to generate a simple traffic matrix in an ASCII file or to generate a PNG image of the traffic pattern \\
\hline
normalization \paramType{long} & Ignored & Positive int & Determines a normalization value that will effect the color scales of an output PNG file. By default, if not specified, the spyplot will determine the largest value and normalize all values for the output PNG to be 0.0-1.0. Values larger than 1.0 will have the same color as 1.0. Useful for having two PNG files with different values have the same color scales. \\
\hline
\end{tabular}

\subsubsection{Namespace ``node.nic.local\_bytes\_sent''}
\openTable
\hline
fileroot \paramType{string} & No default & & 
  The path of a file to be created (or overwritten) where results from the statistics will be placed. 
 \\
\hline
\end{tabular}

\subsubsection{Namespace ``node.nic.global\_bytes\_sent''}
\openTable
\hline
fileroot \paramType{string} & No default & & 
  The path of a file to be created (or overwritten) where results from the statistics will be placed. 
 \\
\hline
\end{tabular}

\subsubsection{Namespace ``node.nic.message\_size\_histogram''}
\openTable
\hline
fileroot \paramType{string} & No default & & 
  The path of a file to be created (or overwritten) where results from the statistics will be placed. 
 \\
\hline
bin\_size \paramType{quantity} & No default & Positive value & For the quantity being collected (bytes,time,etc), this determines how large a value range should be binned together. \\
\hline
num\_bins \paramType{int} & 20 & Positive int & Determines the number of initial bins to allocate. If number is too small, simulation will allocate more. This is just a guess for optimization. \\
\hline
logarithmic \paramType{bool} & false & & Whether to use a logarithmic axis for binning. \\
\hline
\end{tabular}

\subsubsection{Namespace ``node.nic.injection"}
\openTable
\hline
arbitrator & string & cut\_through & null, simple, cut\_through & Determines the type of bandwidth arbitrator for PISCES congestion modeling. Null arbitrator computes simple delays assuming no congestion. Simple uses a basic store-and-forward mechanism that is cheaper to compute, but can have severe latency errors for large packets. Cut-through uses flow approximations to model pipelining of flits across routers and stages.  \\
\hline
latency & time & No default & & If given, this parameter overwrites the send and credit latency parameters. Depending on the type of component, the entire latency will be put on either the credits or the send latency.  For example, crossbars send to a local output buffer but send credits across a network link to a neighbor router. Thus the latency is assume to be entirely on credits with zero latency for forwarding to the local buffer. \\
\hline
bandwidth & bandwidth & No default & & The bandwidth of the arbitrator \\
\hline
send\_latency & time & No default & & The latency to send a packet to the next stage in the network. This can be omitted if the generic latency parameter is given (see above). \\
\hline
credit\_latency & time & No default & & The latency to send a credit to the previous network stage. This can be omitted if the generic latency parameter is given (see above). \\
\hline
num\_vc & int & Computed & Positive int & If not specified, the system will try to compute the correct number of virtual channels based on the topology and routing algorithm. If give and parameter is too small, the system can deadlock. If too large, buffer resources will be underutilized. \\
\hline
credits & byte length & No default & & The number of initial credits for the component. This corresponds to the size of an input buffer on another component. In many cases, if unspecified, SST/macro can compute this from other parameters and fill in the value. In some cases, it will be required. \\
\hline
mtu & byte length & 1024B & & The packet size. All messages (flows) will be broken into units of this size. \\


\hline
\end{tabular}


\subsection{Namespace ``node.netlink''}
\openTable
\hline
num\_inject \paramType{int} & & & \\
\hline
num\_eject \paramType{int} & & & \\
\hline
\end{tabular}
\subsubsection{Namespace ``node.netlink.injection"}
\openTable
\hline
arbitrator & string & cut\_through & null, simple, cut\_through & Determines the type of bandwidth arbitrator for PISCES congestion modeling. Null arbitrator computes simple delays assuming no congestion. Simple uses a basic store-and-forward mechanism that is cheaper to compute, but can have severe latency errors for large packets. Cut-through uses flow approximations to model pipelining of flits across routers and stages.  \\
\hline
latency & time & No default & & If given, this parameter overwrites the send and credit latency parameters. Depending on the type of component, the entire latency will be put on either the credits or the send latency.  For example, crossbars send to a local output buffer but send credits across a network link to a neighbor router. Thus the latency is assume to be entirely on credits with zero latency for forwarding to the local buffer. \\
\hline
bandwidth & bandwidth & No default & & The bandwidth of the arbitrator \\
\hline
send\_latency & time & No default & & The latency to send a packet to the next stage in the network. This can be omitted if the generic latency parameter is given (see above). \\
\hline
credit\_latency & time & No default & & The latency to send a credit to the previous network stage. This can be omitted if the generic latency parameter is given (see above). \\
\hline
num\_vc & int & Computed & Positive int & If not specified, the system will try to compute the correct number of virtual channels based on the topology and routing algorithm. If give and parameter is too small, the system can deadlock. If too large, buffer resources will be underutilized. \\
\hline
credits & byte length & No default & & The number of initial credits for the component. This corresponds to the size of an input buffer on another component. In many cases, if unspecified, SST/macro can compute this from other parameters and fill in the value. In some cases, it will be required. \\
\hline
mtu & byte length & 1024B & & The packet size. All messages (flows) will be broken into units of this size. \\


\hline
\end{tabular}


\subsubsection{Namespace ``node.netlink.ejection"}
\openTable
\hline
arbitrator & string & cut\_through & null, simple, cut\_through & Determines the type of bandwidth arbitrator for PISCES congestion modeling. Null arbitrator computes simple delays assuming no congestion. Simple uses a basic store-and-forward mechanism that is cheaper to compute, but can have severe latency errors for large packets. Cut-through uses flow approximations to model pipelining of flits across routers and stages.  \\
\hline
latency & time & No default & & If given, this parameter overwrites the send and credit latency parameters. Depending on the type of component, the entire latency will be put on either the credits or the send latency.  For example, crossbars send to a local output buffer but send credits across a network link to a neighbor router. Thus the latency is assume to be entirely on credits with zero latency for forwarding to the local buffer. \\
\hline
bandwidth & bandwidth & No default & & The bandwidth of the arbitrator \\
\hline
send\_latency & time & No default & & The latency to send a packet to the next stage in the network. This can be omitted if the generic latency parameter is given (see above). \\
\hline
credit\_latency & time & No default & & The latency to send a credit to the previous network stage. This can be omitted if the generic latency parameter is given (see above). \\
\hline
num\_vc & int & Computed & Positive int & If not specified, the system will try to compute the correct number of virtual channels based on the topology and routing algorithm. If give and parameter is too small, the system can deadlock. If too large, buffer resources will be underutilized. \\
\hline
credits & byte length & No default & & The number of initial credits for the component. This corresponds to the size of an input buffer on another component. In many cases, if unspecified, SST/macro can compute this from other parameters and fill in the value. In some cases, it will be required. \\
\hline
mtu & byte length & 1024B & & The packet size. All messages (flows) will be broken into units of this size. \\


\hline
\end{tabular}


\subsection{Namespace ``node.memory''}
\openTable
\hline
arbitrator (string) & cut\_through & null, simple, cut\_through & The type of arbitrator. See arbitrator descriptions above. \\
\hline
latency (time) & No default & & The latency of single memory operation \\
\hline
total\_bandwidth & No default &  & The total memory bandwidth possible across all memory controllers. \\
\hline
max\_single\_bandwidth & Computed & & The maximum memory bandwidth of a single stream of requests. Essentially the bandwidth of a single memory controller. If not given, this defaults the value of total\_bandwidth. \\
\hline
\end{tabular}
\subsection{Namespace ``node.os"}
\openTable
\hline
stack\_size \paramType{byte length} & 32KB & & The size of user-space thread stack to allocate for each virtual application \\
\hline
stack\_protect \paramType{bool} & false & & Whether to put special mprotect regions around the thread stacks to turn stack overflow errors into seg faults \\
\hline
stack\_chunk\_size \paramType{byte length} & 1MB & & The size of memory to allocate at a time when allocating new thread stacks. Rather than allocating one thread stack at a time, multiple stacks are allocated and added to a pool as needed.  \\
\hline
\end{tabular}
\subsubsection{Namespace ``node.os.call\_graph"}
\openTable
\hline
fileroot \paramType{string} & No default & & 
  The path of a file to be created (or overwritten) where results from the statistics will be placed. 
 \\
\hline
\end{tabular}
\subsubsection{Namespace ``node.os.ftq"}
\openTable
\hline
fileroot \paramType{string} & No default & & 
  The path of a file to be created (or overwritten) where results from the statistics will be placed. 
 \\
\hline 
epoch \paramType{time} & No default & & The length of simulation time to treat as a single interval. This is much like a histogram bin, except the bin is partitioned into multiple categories. If too small, not enough data will be collected per interval to give reasonable looking results. If too large, significant changes in system activity over time will not be resolved. \\
\hline
\end{tabular}
\subsection{Namespace ``node.proc''}
\openTable
\hline
ncores \paramType{int} & No default & Positive int & The number of cores contained in a processor (socket). Total number of cores for a node is $ncores \times nsockets$. \\
\hline
frequency & No default & & The baseline frequency of the node \\
\hline
parallelism & double \\
\hline
\end{tabular}

\section{Namespace ``mpi"}
\openTable
\hline
test\_delay \paramType{time} & 0 & & The minimum time spent by MPI on each MPI\_Test call \\
\hline
iprobe\_delay \paramType{time} & 0 & & The minimum time spent by MPI on each MPI\_Iprobe call \\
\hline
\end{tabular}
\subsection{Namespace ``mpi.queue''}
\openTable
\hline
max\_vshort\_msg\_size \paramType{byte length} & 512B & & The maximum size to use the very short message protocol. Small messages are sent eagerly using special pre-allocated mailbox buffers. Sends complete immediately. \\
\hline
max\_eager\_msg\_size \paramType{byte length} & 8KB & & The maximum size to use the RDMA eager protocol. This also uses buffers to send message, but instead of using pre-allocated mailboxes, it coordinates an RDMA get. Sends complete immediately. \\
\hline
post\_rdma\_delay \paramType{time} & 0 & & The minimum time spent by MPI posting each RDMA operation \\
\hline
post\_header\_delay \paramType{time} & 0 & & The mimimum time spent by MPI sending headers into pre-allocated mailboxes \\
\hline
poll\_delay (time) & 0 & & The minimum time spent by MPI each time it polls for incoming messages \\
\hline
\end{tabular}

\section{Namespace ``switch''}
\openTable
\hline
buffer\_size \paramType{byte length} & No default & & The size of input and output buffers on each switch. This determines the number of credits available to other components \\
\hline
\end{tabular}
\subsection{Namespace ``switch.router''}
\openTable
\hline
ugal\_threshold \paramType{int} & 0 & & The minimum number of network hops required before UGAL is considered. All path lengths less than value automatically use minimal. \\
\hline
\end{tabular}
\subsection{Namespace ``switch.output\_buffer"}
\openTable
\hline
stats & string & null & bytes\_sent, 
	congestion\_spyplot,
	congestion\_delay,
	bytes\_sent,
	byte\_hops,
	delay\_histogram,
	null
  & The  \\
\hline
\end{tabular}
\subsubsection{Namespace ``switch.output\_buffer.delay\_histogram''}
\openTable
\hline
fileroot \paramType{string} & No default & & 
  The path of a file to be created (or overwritten) where results from the statistics will be placed. 
 \\
\hline
bin\_size \paramType{quantity} & No default & Positive value & For the quantity being collected (bytes,time,etc), this determines how large a value range should be binned together. \\
\hline
num\_bins \paramType{int} & 20 & Positive int & Determines the number of initial bins to allocate. If number is too small, simulation will allocate more. This is just a guess for optimization. \\
\hline
logarithmic \paramType{bool} & false & & Whether to use a logarithmic axis for binning. \\
\hline
\end{tabular}

\subsubsection{Namespace ``switch.output\_buffer.congestion\_spyplot''}
\openTable
\hline
fileroot \paramType{string} & No default & & 
  The path of a file to be created (or overwritten) where results from the statistics will be placed. 
 \\
\hline 
type \paramType{string} & ascii & ascii, png & Whether to generate a simple traffic matrix in an ASCII file or to generate a PNG image of the traffic pattern \\
\hline
normalization \paramType{long} & Ignored & Positive int & Determines a normalization value that will effect the color scales of an output PNG file. By default, if not specified, the spyplot will determine the largest value and normalize all values for the output PNG to be 0.0-1.0. Values larger than 1.0 will have the same color as 1.0. Useful for having two PNG files with different values have the same color scales. \\
\hline
\end{tabular}

\subsubsection{Namespace ``switch.output\_buffer.bytes\_sent''}
\openTable
\hline
fileroot \paramType{string} & No default & & 
  The path of a file to be created (or overwritten) where results from the statistics will be placed. 
 \\
\hline
\end{tabular}

\subsubsection{Namespace ``switch.output\_buffer.byte\_hops''}
\openTable
\hline
fileroot \paramType{string} & No default & & 
  The path of a file to be created (or overwritten) where results from the statistics will be placed. 
 \\
\hline
\end{tabular}


\subsection{Namespace ``switch.xbar"}
\openTable
\hline
stats & string & null & bytes\_sent, 
	congestion\_spyplot,
	congestion\_delay,
	bytes\_sent,
	byte\_hops,
	delay\_histogram,
	null
  & The  \\
\hline
arbitrator & string & cut\_through & null, simple, cut\_through & Determines the type of bandwidth arbitrator for PISCES congestion modeling. Null arbitrator computes simple delays assuming no congestion. Simple uses a basic store-and-forward mechanism that is cheaper to compute, but can have severe latency errors for large packets. Cut-through uses flow approximations to model pipelining of flits across routers and stages.  \\
\hline
latency & time & No default & & If given, this parameter overwrites the send and credit latency parameters. Depending on the type of component, the entire latency will be put on either the credits or the send latency.  For example, crossbars send to a local output buffer but send credits across a network link to a neighbor router. Thus the latency is assume to be entirely on credits with zero latency for forwarding to the local buffer. \\
\hline
bandwidth & bandwidth & No default & & The bandwidth of the arbitrator \\
\hline
send\_latency & time & No default & & The latency to send a packet to the next stage in the network. This can be omitted if the generic latency parameter is given (see above). \\
\hline
credit\_latency & time & No default & & The latency to send a credit to the previous network stage. This can be omitted if the generic latency parameter is given (see above). \\
\hline
num\_vc & int & Computed & Positive int & If not specified, the system will try to compute the correct number of virtual channels based on the topology and routing algorithm. If give and parameter is too small, the system can deadlock. If too large, buffer resources will be underutilized. \\
\hline
credits & byte length & No default & & The number of initial credits for the component. This corresponds to the size of an input buffer on another component. In many cases, if unspecified, SST/macro can compute this from other parameters and fill in the value. In some cases, it will be required. \\
\hline
mtu & byte length & 1024B & & The packet size. All messages (flows) will be broken into units of this size. \\


\hline
\end{tabular}

\subsubsection{Namespace ``switch.xbar.delay\_histogram''}
\openTable
\hline
fileroot \paramType{string} & No default & & 
  The path of a file to be created (or overwritten) where results from the statistics will be placed. 
 \\
\hline
bin\_size \paramType{quantity} & No default & Positive value & For the quantity being collected (bytes,time,etc), this determines how large a value range should be binned together. \\
\hline
num\_bins \paramType{int} & 20 & Positive int & Determines the number of initial bins to allocate. If number is too small, simulation will allocate more. This is just a guess for optimization. \\
\hline
logarithmic \paramType{bool} & false & & Whether to use a logarithmic axis for binning. \\
\hline
\end{tabular}

\subsubsection{Namespace ``switch.xbar.congestion\_spyplot''}
\openTable
\hline
fileroot \paramType{string} & No default & & 
  The path of a file to be created (or overwritten) where results from the statistics will be placed. 
 \\
\hline 
type \paramType{string} & ascii & ascii, png & Whether to generate a simple traffic matrix in an ASCII file or to generate a PNG image of the traffic pattern \\
\hline
normalization \paramType{long} & Ignored & Positive int & Determines a normalization value that will effect the color scales of an output PNG file. By default, if not specified, the spyplot will determine the largest value and normalize all values for the output PNG to be 0.0-1.0. Values larger than 1.0 will have the same color as 1.0. Useful for having two PNG files with different values have the same color scales. \\
\hline
\end{tabular}

\subsubsection{Namespace ``switch.xbar.bytes\_sent''}
\openTable
\hline
fileroot \paramType{string} & No default & & 
  The path of a file to be created (or overwritten) where results from the statistics will be placed. 
 \\
\hline
\end{tabular}

\subsubsection{Namespace ``switch.xbar.byte\_hops''}
\openTable
\hline
fileroot \paramType{string} & No default & & 
  The path of a file to be created (or overwritten) where results from the statistics will be placed. 
 \\
\hline
\end{tabular}

\subsection{Namespace ``switch.link''}
\openTable
\hline
arbitrator & string & cut\_through & null, simple, cut\_through & Determines the type of bandwidth arbitrator for PISCES congestion modeling. Null arbitrator computes simple delays assuming no congestion. Simple uses a basic store-and-forward mechanism that is cheaper to compute, but can have severe latency errors for large packets. Cut-through uses flow approximations to model pipelining of flits across routers and stages.  \\
\hline
latency & time & No default & & If given, this parameter overwrites the send and credit latency parameters. Depending on the type of component, the entire latency will be put on either the credits or the send latency.  For example, crossbars send to a local output buffer but send credits across a network link to a neighbor router. Thus the latency is assume to be entirely on credits with zero latency for forwarding to the local buffer. \\
\hline
bandwidth & bandwidth & No default & & The bandwidth of the arbitrator \\
\hline
send\_latency & time & No default & & The latency to send a packet to the next stage in the network. This can be omitted if the generic latency parameter is given (see above). \\
\hline
credit\_latency & time & No default & & The latency to send a credit to the previous network stage. This can be omitted if the generic latency parameter is given (see above). \\
\hline
num\_vc & int & Computed & Positive int & If not specified, the system will try to compute the correct number of virtual channels based on the topology and routing algorithm. If give and parameter is too small, the system can deadlock. If too large, buffer resources will be underutilized. \\
\hline
credits & byte length & No default & & The number of initial credits for the component. This corresponds to the size of an input buffer on another component. In many cases, if unspecified, SST/macro can compute this from other parameters and fill in the value. In some cases, it will be required. \\
\hline
mtu & byte length & 1024B & & The packet size. All messages (flows) will be broken into units of this size. \\


\hline
\end{tabular}

\subsection{Namespace ``switch.ejection''}
\openTable
\hline
arbitrator & string & cut\_through & null, simple, cut\_through & Determines the type of bandwidth arbitrator for PISCES congestion modeling. Null arbitrator computes simple delays assuming no congestion. Simple uses a basic store-and-forward mechanism that is cheaper to compute, but can have severe latency errors for large packets. Cut-through uses flow approximations to model pipelining of flits across routers and stages.  \\
\hline
latency & time & No default & & If given, this parameter overwrites the send and credit latency parameters. Depending on the type of component, the entire latency will be put on either the credits or the send latency.  For example, crossbars send to a local output buffer but send credits across a network link to a neighbor router. Thus the latency is assume to be entirely on credits with zero latency for forwarding to the local buffer. \\
\hline
bandwidth & bandwidth & No default & & The bandwidth of the arbitrator \\
\hline
send\_latency & time & No default & & The latency to send a packet to the next stage in the network. This can be omitted if the generic latency parameter is given (see above). \\
\hline
credit\_latency & time & No default & & The latency to send a credit to the previous network stage. This can be omitted if the generic latency parameter is given (see above). \\
\hline
num\_vc & int & Computed & Positive int & If not specified, the system will try to compute the correct number of virtual channels based on the topology and routing algorithm. If give and parameter is too small, the system can deadlock. If too large, buffer resources will be underutilized. \\
\hline
credits & byte length & No default & & The number of initial credits for the component. This corresponds to the size of an input buffer on another component. In many cases, if unspecified, SST/macro can compute this from other parameters and fill in the value. In some cases, it will be required. \\
\hline
mtu & byte length & 1024B & & The packet size. All messages (flows) will be broken into units of this size. \\


\hline
\end{tabular}


\section{Namespace ``appN''}
This is a series of namespaces \inlineshell{app1}, \inlineshell{app2}, and so on for each of the launched applications.

\openTable
\hline
name \paramType{string} & No default & See starting in Section \ref{sec:tutorial:basicmpi} & The name of the application to launch \\
\hline
size \paramType{int} & No default & Positive int & The number of procs (MPI ranks) to launch. If launch\_cmd given, this parameter is not required. \\
\hline
start \paramType{int} & 0 & & The time at which a launch request for the application will be made \\
\hline
concentration \paramType{int} & 1 & Positive int & The number of procs (MPI ranks) per compute node \\
\hline
core\_affinities \paramType{vector of int} & Empty & & \\
\hline
launch\_cmd \paramType{string} & No default & Valid aprun or srun & This uses a launch command as would be found with ALPS or SLURM launchers on real systems, e.g. aprun -n 4 -N 1 \\
\hline
launch\_indexing \paramType{string} & block & block, random, cart, node\_id, coordinate & The indexing scheme for assign proc ID (MPI rank number) to compute nodes \\
\hline
launch\_node\_id\_mapper\_file (filepath) & No default & & If using Node ID indexing, the file containing the node ID index list \\
\hline 
random\_indexer\_seed \paramType{long} & System time & & The seed to use for a random allocation. If not specified, system time is used. \\
\hline
launch\_allocation \paramType{string} & first\_available & first\_available, random, cart, node\_id, coordinate & The scheme to use for allocating compute nodes to a given job. \\
\hline
random\_allocation\_seed \paramType{long} & System time & & For random allocation policy. If unspecified, system time is used as the seed.  \\
\hline
launch\_node\_id\_allocation\_file \paramType{filepath} & No default & & If using Node ID allocation, the file containing the list of node IDs to allocate for the job \\
\hline
launch\_dumpi\_metaname \paramType{filepath} & No default & & If running DUMPI trace, the location of the metafile for configuring trace replay \\ 
\hline
launch\_coordinate\_file \paramType{filepath} & No default & & If running using coordinate allocation or indexing, the path to the file containing the node coordinates of each proc (MPI rank) \\ 
\hline
cart\_launch\_sizes \paramType{vector of int} & No default & & Launch a contiguous block of nodes in a Cartesian topology. This gives the size of each dimension in the block. \\
\hline 
cart\_launch\_offsets \paramType{vector of int} & No default & & Launch a contiguous block nodes in a Cartesian topology. This gives the offset in each dimension where the block begins. \\
\hline
parsedumpi\_timescale \paramType{double} & 1.0 & Positive float & If running DUMPI traces, scale compute times by the given value. Values less than 1.0 speed up computation. Values greater than 1.0 slow down computation. \\
\hline
parsedumpi\_terminate\_percent \paramType{int} & 100 & 1-100 & Percent of trace. Can be used to terminate large traces early \\
\hline
\end{tabular}



