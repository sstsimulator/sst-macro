%% !TEX root = manual.tex

\newcommand{\openTable}{\begin{tabular}{| c | c | p{2cm} | p{2cm} | p{6.5cm} |}}

\chapter{Detailed Parameter Listings}
\label{chapter:parameters}
The following chapter is organized by parameter namespaces.

Tables in each namespace are organized as
\def\arraystretch{1.5}%  1 is the default, change whatever you need

\openTable
\hline
Name & Type & Required/ Optional & Allowed \newline Values & Description \\
\hline
\end{tabular}

which lists the possible parameter names, allowed values, and brief descriptions.
Where required, more detail descriptions of particular parameter values are shown.

The allow parameter types are:

\begin{tabular}{l | l}
\hline
int & Any integer \\
\hline
long & Any integer value, but guaranteed not to overflow for long integers \\
\hline
freq & Any valid float value followed by frequency units (Hz, MHz, GHz) \\
\hline
bandwidth & Any valid float value followed by bandwidth units (b/s, B/s, Mb/s, MB/s, etc) \\
\hline
byte length & Any positive integer followed by length units (B, KB, MB, GB,TB) \\
\hline
string & An arbitrary string \\
\hline
vector of X & A vector of type X with entries separated by spaces \\
\hline
\end{tabular}
\section{Global namespace}
\openTable
\hline
sst\_nthread & int & Optional, default 1 & Any positive integer & Specifying more threads than cores can lead to deadlock \\
\hline
serialization\_buffer\_size & byte length \\
\hline
mpi\_max\_num\_requests & int \\
\hline
cpu\_affinity & vector of int \\
\hline
\end{tabular}

\section{Namespace ``topology''}
\openTable
\hline
geometry & \\
\hline
name & \\
\hline
\end{tabular}

\section{Namespace ``node''}
\openTable
\hline
services & vector of strings \\
\hline
\end{tabular}
\subsection{Namespace ``node.nic''}
\openTable
\hline
negligible\_size & byte length \\
\hline
\end{tabular}
\subsubsection{Namespace ``node.nic.traffic\_matrix''}
\subsubsection{Namespace ``node.nic.local\_bytes\_sent''}
\subsubsection{Namespace ``node.nic.global\_bytes\_sent''}
\subsubsection{Namespace ``node.nic.message\_size\_histogram''}
\subsubsection{Namespace ``node.nic.injection"}
\openTable
\hline
arbitrator & string & \\
\hline
latency & time & \\
\hline
bandwidth & bandwidth & \\
\hline
send\_latency & time & \\
\hline
credit\_latency & time & \\
\hline
credits & byte length \\
\hline
\end{tabular}

\subsection{Namespace ``node.netlink''}
\subsubsection{Namespace ``node.netlink.injection"}
\openTable
\hline
arbitrator & string & \\
\hline
latency & time & \\
\hline
bandwidth & bandwidth & \\
\hline
send\_latency & time & \\
\hline
credit\_latency & time & \\
\hline
credits & byte length \\
\hline
\end{tabular}

\subsubsection{Namespace ``node.netlink.ejection"}
\openTable
\hline
arbitrator & string & \\
\hline
latency & time & \\
\hline
bandwidth & bandwidth & \\
\hline
send\_latency & time & \\
\hline
credit\_latency & time & \\
\hline
credits & byte length \\
\hline
\end{tabular}

\subsection{Namespace ``node.memory''}
\openTable
\hline
arbitrator & string & \\
\hline
latency & time & \\
\hline
total\_bandwidth & bandwidth & \\
\hline
max\_single\_bandwidth & bandwidth & \\
\hline
\end{tabular}
\subsection{Namespace ``node.os"}
\openTable
\end{tabular}
\subsubsection{Namespace ``node.os.call\_graph"}
\subsubsection{Namespace ``node.os.ftq"}
\subsection{Namespace ``node.proc''}
\openTable
\hline
ncores & int & Required & Any positive integer & The number of cores contained in a processor (socket). Total number of cores for a node is $ncores \times nsockets$. \\
\hline
frequency & freq & Required & & The baseline frequency of the node \\
\hline
parallelism & double \\
\hline
\end{tabular}

\section{Namespace ``mpi"}
\openTable
\end{tabular}
\subsection{Namespace ``mpi.queue''}
\openTable
\end{tabular}

\section{Namespace ``switch''}
\openTable
\hline
buffer\_size & byte length \\
\hline
\end{tabular}
\subsection{Namespace ``switch.output\_buffer"}
\subsection{Namespace ``switch.xbar"}
\openTable
\hline
arbitrator & string & \\
\hline
latency & time & \\
\hline
bandwidth & bandwidth & \\
\hline
send\_latency & time & \\
\hline
credit\_latency & time & \\
\hline
num\_vc & int \\
\hline
credits & byte length \\
\hline
\end{tabular}
\subsubsection{Namespace ``switch.xbar.delay\_histogram''}
\subsubsection{Namespace ``switch.xbar.congestion\_spyplot''}
\subsubsection{Namespace ``switch.xbar.bytes\_sent''}
\subsubsection{Namespace ``switch.xbar.byte\_hops''}
\subsection{Namespace ``switch.link''}
\openTable
\hline
arbitrator & string & \\
\hline
latency & time & \\
\hline
bandwidth & bandwidth & \\
\hline
send\_latency & time & \\
\hline
credit\_latency & time & \\
\hline
credits & byte length \\
\hline
\end{tabular}
\subsubsection{Namespace ``switch.link.delay\_histogram''}
\subsubsection{Namespace ``switch.link.congestion\_spyplot''}
\subsubsection{Namespace ``switch.link.bytes\_sent''}
\subsubsection{Namespace ``switch.link.byte\_hops''}
\subsection{Namespace ``switch.ejection''}
\openTable
\hline
arbitrator & string & \\
\hline
latency & time & \\
\hline
bandwidth & bandwidth & \\
\hline
send\_latency & time & \\
\hline
credit\_latency & time & \\
\hline
credits & byte length \\
\hline
\end{tabular}

\section{Namespace ``appN''}
\openTable
\hline
launch\_cmd & \\
\hline
launch\_indexing & \\
\hline
launch\_allocation & \\
\hline
\end{tabular}
\paragraph{aprun}
\paragraph{srun}



