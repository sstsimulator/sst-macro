%% !TEX root = manual.tex

\chapter{Topologies}
\label{chapter:topologies}


The torus topology is straightforward and easy to understand.
Here we introduce the basics of other topologies within SST that are more complex and require extra documentation to configure properly.
These are generally higher-radix or path-diverse topologies like fat tree, dragonfly, and flattened butterfly.  
As noted in \ref{sec:tutorial:topology}, a more thorough and excellent discussions of these topologies is given in ``High Performance Datacenter Networks'' by Dennis Abts and John Kim.

\section{Topology Query Utility}
Understanding topology inputs and geometries can sometimes be challenging.
\sstmacro provides an executable for testing topology inputs and doing example coordinate computations.
After making and installing, an executable \inlineshell{sstmac_top_info} will appear in the \inlineshell{bin} folder.
The invocation of \inlineshell{sstmac_top_info} is exactly the same as the main \inlineshell{sstmac} executable.
For the example parameter file named \inlineshell{machine.ini}:

\begin{ViFile}
topology.name = fattree
topology.geometry = 4 3
\end{ViFile}

we run

\begin{ShellCmd}
bin> sstmac_top_info -f machine.ini
\end{ShellCmd}
which produces the output

\begin{ViFile}
Number of nodes:         81
Number of leaf switches: 27
Number of switches:      94
\end{ViFile}

detailing the produced geometry.  Here the fat tree has a total of 94 switches, 27 of which are ``leaf'' switches directly connected to compute nodes.
The output is followed by the prompt

\begin{ShellCmd}
NextInput: 
\end{ShellCmd}

One can either enter a single number (switch ID) or set of coordinates.
If given a switch ID, the coordinates are computed.
If coordinates are given, the switch ID is computed.

\begin{ShellCmd}
NextInput: 32
Switch ID maps to coordinates [ 2 0 1 2 ]
NextInput: 2 0 1 2
Coordinates map to switch ID 32
\end{ShellCmd}

The program is just exited with Ctrl-C.
The meaning of the above coordinates is detail below for fat tree (Section \ref{sec:tutorial:fattree}).


\input{Torus}
\input{Hypercube}
%\input{Butterfly}
\input{FatTree}
\input{Dragonfly}


