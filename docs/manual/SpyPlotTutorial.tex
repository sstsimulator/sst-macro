
%% !TEX root = manual.tex

\section{Spyplot Diagrams}
\label{sec:tutorials:spyplot}

Spyplots visualize communication matrices, showing either the number of messages or number of bytes sent between two network endpoints.
They are essentially contour diagrams, where instead of a continuous function $F(x,y)$ we are plotting the communication matrix $M(i,j)$.
An example spyplot is shown for a simple application that only executes an MPI\_Allreduce (Figure \ref{fig:spyplot}).
Larger amounts of data (red) are sent to nearest neighbors while decreasing amounts (blue) are sent to MPI ranks further away.

\begin{figure}[h]
\centering
\includegraphics[width=0.4\textwidth]{figures/spyplot/mpi_spyplot.png}
\caption{Spyplot of Bytes Transferred Between MPI Ranks for MPI\_Allreduce}
\label{fig:spyplot}
\end{figure}

Various spyplots can be activated by boolean parameters in the input file.
The most commonly used are the MPI spyplots, for which you must add

\begin{ViFile}
mpi_spyplot = <fileroot>
\end{ViFile}

After running there will be a .csv and .png file in the folder, with e.g. \inlineshell{fileroot = test}

\begin{ShellCmd}
example> ls 
test.png
test.csv
\end{ShellCmd}
\inlineshell{test.png} shows the number of bytes exchanged between MPI ranks.
To extend the analysis you can specify

\begin{ViFile}
network_spyplot = <fileroot>
\end{ViFile}
A new csv/png will appear showing the number of bytes exchanged between physical nodes, 
accumulating together all MPI ranks sharing the same node.
This gives a better sense of spatial locality when many MPI ranks are on the same node.


