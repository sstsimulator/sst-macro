% !TEX root = developer.tex


\chapter{Discrete Event Simulation}
\label{chapter:des}
There are abundant tutorials on discrete event simulation around the web.
To understand the basic control flow of \sstmacro simulations,
you should consult Section 3.6, Discrete Event Simulation, in the user's manual.
For here, it suffices to simply understand that objects schedule events to run at a specific time.
When an event runs, it can create new events in the future.
A simulation driver gradually progresses time, running events when their time stamp is reached.
As discussed in the user's manual, we must be careful in the vocabulary.
\emph{Simulation time} or \emph{simulated time} is the predicted time discrete events are happening in the simulated hardware.
\emph{Wall time} or \emph{wall clock time} is the time \sstmacro itself has been running.
There are a variety of classes the cooperate in driving the simulation, which we now describe.

\section{Event Managers}
\label{sec:eventManagers}
The driver for simulations is an event manager that provides the function

\begin{CppCode}
virtual void schedule(TimeDelta start_time, Event* event) = 0;
\end{CppCode}
This function must receive events and order them according to timestamp.
Two main types of data structures can be used, which we briefly describe below.

The event manager also needs to provide

\begin{CppCode}
virtual void run() = 0;
\end{CppCode}

The termination condition can be:
\begin{itemize}
\item A termination timestamp is passed.  For example, a simulation might be specified to terminate after 100 simulation seconds. 
Any pending events after 100 seconds are ignored and the simulation exits.
\item The simulation runs out of events.  With nothing left to do, the simulation exits.
\end{itemize}

Events are stored in a queue (sorted by time)

\begin{CppCode}
namespace sstmac {

class ExecutionEvent
{
 public:
  virtual void execute() = 0;

  ...
};
\end{CppCode}

The execute function is invoked by the \evmgr to run the underlying event.
There are generally two basic event types in \sstmacro, which we now introduce.

\subsection{Event Handlers}
\label{subsec:eventHandlers}
In most cases, the event is represented as an event sent to an object called an \evhandler at a specific simulation time.
In handling the event, the event handlers change their internal state and may cause more events
by scheduling new events at other event handlers (or scheduling messages to itself) at a future time.

In most cases, events are created by calling the function

\begin{CppCode}
auto* ev = newCallback(this, &Actor::act);
\end{CppCode}

This then creates a class of type \inlinecode{ExecutionEvent}, for which the execute function is

\begin{CppCode}
template <int ...S> void dispatch(seq<S...>){
  (obj_->*fxn_)(std::get<S>(params_)...);
}

Fxn fxn_;
Cls* obj_;
std::tuple<Args...> params_;
\end{CppCode}

For example, given a class \inlinecode{Actor} with the member function \inlinecode{act}

\begin{CppCode}
void Actor::Actor(Event* ev, int ev_id){...}
\end{CppCode}
we can create an event handler

\begin{CppCode}
Actor* a = ....;
auto* ev = newCallback(a, &actor::act, 42);
schedule(time, , ev);
\end{CppCode}
When the time arrives for the event, the member function will be invoked

\begin{CppCode}
a->act(42);
\end{CppCode}


\subsection{Event Heap/Map}
\label{subsec:eventHeap}
The major distinction between different event containers is the data structured used.
The simplest data structure is an event heap or ordered event map.
The event manager needs to always be processing the minimum event time, which maps naturally onto a min-heap structure.
Insertion and removal are therefore log(N) operations where N is the number of currently scheduled events.
For most cases, the number and length of events is such that the min-heap is fine.

\section{Event Schedulers}
\label{sec:eventSchedulers}
The simulation is partitioned into objects that are capable of scheduling events.
Common examples of \evscheduler objects are nodes, NICs, memory systems, or the operating system.
In serial runs, an event scheduler is essentially just a wrapper for the \evmgr and the class is not strictly necessary.
There are two types of event scheduler: \inlinecode{Component} and \inlinecode{SubComponent}.
In parallel simulation, though, the simulation must be partitioned into different scheduling units.
Scheduling units are then distributed amongst the parallel processes.
Components are the basic unit.  Currently only nodes and network switches are components.
All other devices (NIC, memory, OS) are subcomponents that must be linked to a parent component.
Even though components and subcomponents can both schedule events (both inherit from \evscheduler),
all subcomponents must belong to a component.  A subcomponent cannot be separated from its parent component during parallel simulation.



